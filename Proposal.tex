\documentclass[12pt]{article}
\usepackage{amsmath}
\usepackage{graphicx}
\usepackage{hyperref}

\title{Roller Coaster Simulation with Friction}
\author{Chengze Song, Bibek Poudel}
\date{November 2024}

\begin{document}

\maketitle

\section*{Problem Statement}
The project involves simulating a roller coaster system with friction to analyze its motion and stability under various conditions. Roller coaster models are essential in understanding the dynamics and control principles in amusement park ride design, ensuring safety and optimal performance. Our objectives are to implement a realistic friction model, simulate the roller coaster's motion, and analyze the effects of different parameters on its stability and performance. Our \textbf{objectives} are :
\begin{itemize}
    \item Implement a realistic friction model in the roller coaster simulation.
    \item Simulate the roller coaster's motion and analyze its stability and performance.
\end{itemize}

\section*{System Description}
Our system consists of a roller coaster track with a cart that moves along the track. The track is modeled as a series of connected segments with different shapes and slopes. The cart is subject to gravity, friction, and other forces that affect its motion. The friction model is essential for accurately simulating the roller coaster's behavior, as it affects the cart's speed and energy dissipation. The roller coaster's motion is simulated using numerical integration methods to solve the equations of motion. The simulation results are analyzed to understand the roller coaster's stability and performance under different conditions.

\section*{Simulation and Analysis}
We plan to simulate the roller coaster's motion using numerical integration methods such as Euler's method or Runge-Kutta methods. The simulation will involve solving the equations of motion for the cart, including the effects of gravity, friction, and other forces. The friction model will be implemented to account for energy dissipation and the cart's speed changes. The simulation results will be analyzed to understand the roller coaster's behavior, stability, and performance under different conditions.

\section*{Validation}
In the final project, we plan to validate our simulation results in two ways:
\begin{enumerate}
    \item Compare the simulation results with analytical solutions for simple cases to verify the correctness of the implementation.
    \item 
\end{enumerate}

\section*{Conclusion}
We aim to develop a realistic roller coaster simulation with friction to analyze its motion and stability. The project will involve implementing a friction model, simulating the roller coaster's motion, and analyzing its performance under different conditions. The simulation results will provide insights into the roller coaster's behavior and help in understanding the dynamics and control principles in amusement park ride design.

\end{document}